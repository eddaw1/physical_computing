\documentclass[../physical_computing.tex]{subfiles}

\begin{document}

% The asterisk excludes chapter from the table of contents.
\chapter*{Preface}

This book grew out of a class that I teach at The University of Sheffield, in the physics department. I intended the class to be a very practical and lab-driven introduction to programming field programmable gate arrays. Indeed, the class can still be looked upon as exactly that. However, as my interests and research have developed, I have realised that there is much to be learned about physics itself from the work of programming digital circuits, and indeed from pondering the nature and function of those circuits. So, being a curious fellow, I have taken brief forays into the mathematical, computational and yes occasionally even the philosophical implications of the practical material in the class. I hope that makes it more interesting. I also hope that my musings in these areas are correct! I welcome comments and corrections from anyone.

The book is based on a series of lab exercises. The students are each lent a laptop running Linux on which the Xilinx (AMD) software Vivado and Vitis are installed. These packages allow you to develop VHDL and occasionally C code for Xilinx FPGAs. The choice of these tools and hardware was based purely on choices made by the author in his research; other companies use different software and, although VHDL has become a standard hardware description language, Verilog is a perfectly viable alternative that, indeed, has arguably become more common in engineering practice. My choice of Linux was for the simple practical reason that I can turn off automatic operating system updates so that the students are not waiting for these to finish when they are in the lab. Vivado and Vitis run perfectly well in Windows and indeed I used to use Windows in the course.

It has been fun developing a course at the boundary of physics, engineering and computer science. As I often point out to the students, Universities are very conservative institutions; the names of the subjects haven't changed in 100 years! The subjects themselves, however, I would claim, have not only changed but have blurred together. This material is as current as it is useful and powerful. I hope you enjoy what is within these pages !\\

\noindent
Ed Daw, 4\textsuperscript{th} February 2024

\end{document}